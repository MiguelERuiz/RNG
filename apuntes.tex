\documentclass{article}
\usepackage{minted}
\usepackage[spanish]{babel}
\usepackage{hyperref}

\begin{document}

\tableofcontents

\section{Configuración del fichero net.conf}
\begin{minted}{bash}
  redes@RED:~$ (nano|vim.tiny) net.conf # Tu editor favorito dentro de la VM
  # NO deben repetirse las umlX.Y o de lo contrario, dará un error al lanzar el escenario
  defsw br12 uml1.0 uml2.0 # configura la conexión entre dos interfaces
  defsw net1 uml1.1 # configura la conexión para una sola interfaz
  defsw br345 uml3.0 uml4.1 uml5.2 # configura la conexión para tres interfaces
\end{minted}

\section{Preparación del entorno}
\begin{minted}{bash}
  redes@RED:~$ sudo ifovsdel # borra las configuraciones previamente existentes
  redes@RED:~$ sudo ifovsparse net.conf # configura las interfaces para lanzar el entorno
  redes@RED:~$ uml{1..N} # crea un rango N de directorios `umlX'
  redes@RED:~$ lanza {1..N} # crea las uml declaradas anteriormente
\end{minted}

\section{Iniciar sesión en una UML}
\begin{minted}{bash}
  uml1 login: root # entramos siempre como root
  root@uml1# # A partir de aqui, estamos dentro de una sesión bash en la UML1
\end{minted}

\section{Acceder al shell de Quagga}
\begin{minted}{bash}
 root@uml1# vtysh
 uml1# # Aqui ya estamos dentro de una sesión de Quagga
\end{minted}

\section{Acceder a la consola para introducir comandos de configuración}
\begin{minted}{bash}
 uml1# # Lo que está entre corchetes significa que es opcional
 uml1# conf[igure] term[inal]
 uml1(config)# ! Estamos dentro de la configuración de la UML
\end{minted}

\section{Acceder a una interfaz}
\begin{minted}{lexer.py:IOSLexer -x}
 uml1(config)# int[erface] eth0
 uml1(config-if)# # Estamos dentro de la interfaz eth0
\end{minted}

\section{Añadir una IP a la interfaz}
\begin{minted}{bash}
 uml1(config-if)# ip address 10.0.0.1/24
 uml1(config-if)# no shutdown # IMPORTANTE para que la interfaz quede levantada
\end{minted}

\section{Activar retransmisión de paquetesr}
\begin{minted}{bash}
 uml1(config)# # Lo que se le dice a la UML donde se ejecuta el siguiente
 uml1(config)# # comando es que actúe como router (encaminador)
 uml1(config)# ip forwarding
\end{minted}

\section{}
\begin{minted}{bash}
\end{minted}

\section{}
\begin{minted}{bash}
\end{minted}

\section{}
\begin{minted}{bash}
\end{minted}

\section{}
\begin{minted}{bash}
\end{minted}

\section{}
\begin{minted}{bash}
\end{minted}

\section{}
\begin{minted}{bash}
\end{minted}

\section{}
\begin{minted}{bash}
\end{minted}

\section{Configurar los demonios de Quagga}
\begin{minted}{bash}
  root@uml1# (nano|vim.tiny) /etc/quagga/daemons # Accedemos a los demonios de Quagga
\end{minted}

\section{Reiniciar y comprobar los demonios activos de Quagga}
\begin{minted}{bash}
 root@uml1# systemctl restart quagga
 root@uml1# systemctl status quagga
\end{minted}

\section{Atajos utiles de líneas de comandos}
\begin{itemize}
  \item Ctrl+a: Ir al inicio de la línea
  \item Ctrl+e: Ir al final de la línea
  \item Ctrl+w: Borra la palabra que está delante del cursor
  \item Ctrl+\_: Deshace el último cambio (El Ctrl+z de toda la vida)
\end{itemize}





\end{document}
