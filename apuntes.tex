\documentclass{article}
\usepackage{minted}
\usepackage[spanish]{babel}
\usepackage{hyperref}


%%%%%%%%%%%%%%%%%%%%%%%%%%%%%%%%%%%%%%%%%%%%%%%%%%%%%%
% COMIENZA EL DOCUMENTO
%%%%%%%%%%%%%%%%%%%%%%%%%%%%%%%%%%%%%%%%%%%%%%%%%%%%%%

\begin{document}

\tableofcontents
\newpage

\section{Configuración del fichero net.conf}
\begin{minted}{bash}
  redes@RED:~$ (nano|vim.tiny) net.conf # Tu editor favorito dentro de la VM
  # NO deben repetirse las umlX.Y o de lo contrario,
  # dará un error al lanzar el escenario
  defsw br12 uml1.0 uml2.0 # configura la conexión entre dos interfaces
  defsw net1 uml1.1 # configura la conexión para una sola interfaz
  defsw br345 uml3.0 uml4.1 uml5.2 # configura la conexión para tres interfaces
\end{minted}

\section{Preparación del entorno}
\begin{minted}{bash}
  redes@RED:~$ sudo ifovsdel # borra las configuraciones previamente existentes
  redes@RED:~$ sudo ifovsparse net.conf # configura las interfaces para lanzar el entorno
  redes@RED:~$ mkdir uml{1..N} # crea un rango de directorios desde uml1 a umlN
  redes@RED:~$ lanza {1..N} # crea las uml declaradas anteriormente
\end{minted}

\section{Finalización del entorno}
\begin{minted}{bash}
  # Ejecuta un ctrl+alt+del en cada una de las UML (forma suave de matar las UML)
  redes@RED:~$ for i in {1..N} ; do uml\_mconsole uml$i cad; done
  # Mata todos los subprocesos linux de golpe (forma dura de matar las UML)
  # Es el último recurso a utilizar en caso de que el anterior comando no funcione
  redes@RED:~$ pkill linux
\end{minted}

\section{Iniciar sesión en una UML}
\begin{minted}{bash}
  uml1 login: root # entramos siempre como root
  root@uml1# # A partir de aqui, estamos dentro de una sesión bash en la UML1
\end{minted}

\section{Acceder al shell de Quagga}
\begin{minted}{bash}
 root@uml1# vtysh
 uml1# # Aqui ya estamos dentro de una sesión de Quagga
\end{minted}

\section{Acceder a la consola para introducir comandos de configuración}
\begin{minted}{lexer.py:IOSLexer -x}
 uml1# # Lo que está entre corchetes significa que es opcional
 uml1# conf[igure] term[inal]
 uml1(config)# ! Estamos dentro de la configuración de la UML
\end{minted}

\section{Acceder a una interfaz}
\begin{minted}{lexer.py:IOSLexer -x}
 uml1(config)# int[erface] eth0
 uml1(config-if)# # Estamos dentro de la interfaz eth0
\end{minted}

\section{Añadir una IP a la interfaz}
\begin{minted}{lexer.py:IOSLexer -x}
 uml1(config-if)# ip address 10.0.0.1/24
 uml1(config-if)# ipv6 address 2001:db8::50:10/64
 uml1(config-if)# no shutdown # IMPORTANTE para que la interfaz quede levantada
\end{minted}

\section{Activar retransmisión de paquetes}
\begin{minted}{lexer.py:IOSLexer -x}
 uml1(config)# # Lo que se le dice a la UML donde se ejecuta el siguiente
 uml1(config)# # comando es que actúe como router (encaminador)
 uml1(config)# ip[v6] forwarding
\end{minted}

\section{Anuncio de prefijo en IPv6}
\begin{minted}{lexer.py:IOSLexer -x}
 uml1(config-if)# ! Los hosts a los que hayan que hacerles el anuncio del prefijo
 uml1(config-if)# ! solo hay que hacerles un `no shutdown' en la interfaz asociada
 uml1(config-if)# ipv6 nd prefix 2001:db8:101:1::/64
 uml1(config-if)# no ipv6 nd suppress-ra
\end{minted}

\section{Guardar la configuración establecida}
\begin{minted}{lexer.py:IOSLexer -x}
 uml1(config)# write ! Guarda toda la configuración
\end{minted}

\section{Zebra}
\subsection{Interface Commands}
\subsubsection{Standard Commands}
\begin{minted}{lexer.py:IOSLexer -x}
 !!! Estos comandos es necesario estar dentro de la interfaz
 ! Apaga o levanta la interfaz
 [no] shutdown
 ! [no] ip[v6] a[ddress] ADDRESS/PREFIX
\end{minted}

\subsection{Zebra Terminal Mode Commands}
\begin{minted}{lexer.py:IOSLexer -x}
 ! Muestra las rutas actuales que almacena zebra en su base de datos
 show ip[v6] route
 ! Muestra información sobre la interfaz
 show interface [INTERFACE] ! ej: eth0
\end{minted}

\section{RIP}
\subsection{Start RIPd}
\begin{minted}{bash}
 # Activar el demonio ripd en el fichero /etc/quagga/daemons
 ~$ systemctl restart quagga
 ~$ systemctl status quagga
\end{minted}

\subsection{Configuration}
\begin{minted}{lexer.py:IOSLexer -x}
 ! Habilita o deshabilita RIP
 [no] router rip
 ! Fija o no la interfaz RIP habilitada por ifname
 [no] network ifname ! ej.: network eth0
 ! Especifica o no un vecino RIP
 [no] neighbor a.b.c.d
 ! Controla split-horizon en la interfaz. Por defecto activado
 [no] ip split-horizon
\end{minted}

\subsection{Show RIP Information}
\begin{minted}{lexer.py:IOSLexer -x}
 ! Muestra rutas RIP
 show ip rip
 ! Muestra estado actual de RIP
 show ip rip status
\end{minted}

\section{RIPng}
\subsection{Start RIPngd}
\begin{minted}{bash}
 # Activar el demonio ripngd en el fichero /etc/quagga/daemons
 ~$ systemctl restart quagga
 ~$ systemctl status quagga
\end{minted}

\subsection{Configuration}
\begin{minted}{lexer.py:IOSLexer -x}
 ! Habilita o deshabilita RIPng
 [no] router ripng
 ! Fija o no la interfaz RIPng habilitada por ifname
 [no] network ifname ! ej.: network eth0
\end{minted}

\subsection{Show RIPng Information}
\begin{minted}{lexer.py:IOSLexer -x}
 ! Muestra rutas RIPng
 show ip ripng
\end{minted}


\begin{minted}{bash}
\end{minted}

\section{OSPFv2}
\subsection{Configuration}
\begin{minted}{bash}
 # Activar el demonio ospfd en el fichero /etc/quagga/daemons
 ~$ systemctl restart quagga
 ~$ systemctl status quagga
\end{minted}

\subsection{Router}
\begin{minted}{lexer.py:IOSLexer -x}
 ! Activa o desactiva el proceso OSPF
 [no] router ospf
 ! Fija o quit el router-id del proceso OSPF
 [no] ospf router-id a.b.c.d
 ! Solo se anuncia la interfaz como link stub en router LSA
 ! Hay que ponerlo en las redes de usuario
 passive-interface ifname
 ! Habilita o deshabilita stub router advertisement support
 [no] max-metric router-lsa [on-startup|on-shutdown|administrative]
 ! Especifica o no las interfaces habilitadas en OSPF
 [no] network a.b.c.d/m area <0-4294967295> ! también puede ser en el formato a.b.c.d
\end{minted}

\subsection{Area}
\begin{minted}{lexer.py:IOSLexer -x}
 ! Agrega o desagrega rutas inter-área (solo válido para ABR)
 [no] area a.b.c.d range a.b.c.d/m ! Un ejemplo en la práctica de OSPFv2
 ! Define o elimina un área stub
 [no] area a.b.c.d sub [no-summary] 
\end{minted}

\subsection{Showing OSPF Information}
\begin{minted}{lexer.py:IOSLexer -x}
 ! Muestra info de proposito general de OSPF
 show ip ospf
 ! Muestra estado y configuracion de la interfaz especifica OSPF (todas si
 ! no se especifica alguna)
 show ip ospf interface [INTERFACE]
 ! Muestra info de la BBDD OSPF
 show ip ospf database
 ! Muestra la tabla de rutas OSPF
 show ip ospf route
\end{minted}


\section{OSPFv3}
\subsection{Configuration}
\begin{minted}{bash}
 # Activar el demonio ospf6d en el fichero /etc/quagga/daemons
 ~$ systemctl restart quagga
 ~$ systemctl status quagga
\end{minted}

\subsection{Router}
\begin{minted}{lexer.py:IOSLexer -x}
 ! Acceder a la configuracion del router
 router ospf6
 ! Fijar un router-id
 router-id a.b.c.d ! ej: 0.0.0.1
 ! Bindea una interfaz a un area y comienza el envio de paquetes
 interface ifname area a.b.c.d ! ej: interface eth0 area 0.0.0.1
\end{minted}

\subsection{Interface}
\begin{minted}{lexer.py:IOSLexer -x}
 !!! Estos comandos es necesario estar dentro de la interfaz

 ! Establece coste de la interfaz. El valor por defecto depende
 ! del ancho de banda de la interfaz y del coste del ancho de
 ! banda de referencia
 ipv6 opsf6 cost COST ! ej: 2
 ! Establece coste del intervalo HELLO de la interfaz
 ipv6 opsf6 hello-interval HELLOINTERVAL ! ej: 60
 ! Establece intervalo Router Dead en la interfaz
 ipv6 ospf6 dead-interval DEADINTERVAL ! ej: 240
 ! Establece intervalo de retransmisión de la interfaz
 ipv6 ospf6 retransmit-interval RETRANSMITINTERVAL ! ej: 5
 ! Establece Router Priority de la interfaz
 ipv6 opsf6 priority PRIORITY ! ej: 5
 ! Establece Inf-Trans-Delay de la interfaz
 ipv6 ospf6 transmit-delay TRANSMITDELAY ! ej: 1
\end{minted}

\subsection{Showing OSPF Information}
\begin{minted}{lexer.py:IOSLexer -x}
 ! Muestra info acerca de OSPFv3 de la UML
 show ipv6 ospf6 [INSTANCE_ID]
 ! Muestra el resumen de la base de datos LSA
 show ipv6 ospf6 database
 ! Muestra la configuracion de la interfaz OSPF
 show ipv6 ospf6 interface
 ! Muestra el estado y el backup del vecino
 show ipv6 ospf6 neighbor
 ! Muestra la request-list del vecino
 show ipv6 ospf6 request-list A.B.C.D
 ! Muestra la tabla de rutas interna
 show ipv6 route ospf6
\end{minted}

\section{Configurar los demonios de Quagga}
\begin{minted}{bash}
  root@uml1# (nano|vim.tiny) /etc/quagga/daemons # Accedemos a los demonios de Quagga
\end{minted}

\section{Reiniciar y comprobar los demonios activos de Quagga}
\begin{minted}{bash}
 root@uml1# systemctl restart quagga
 root@uml1# systemctl status quagga
\end{minted}

\section{Atajos útiles de líneas de comandos}
\begin{itemize}
  \item Ctrl+a: Ir al inicio de la línea
  \item Ctrl+e: Ir al final de la línea
  \item Ctrl+w: Borra la palabra que está delante del cursor
  \item Ctrl+d: borra el caracter delante del cursor (supr de toda la vida)
  \item Ctrl+\_: Deshace el último cambio (El Ctrl+z de toda la vida)
  \item Ctrl+r: Busca por el historial del shell hacia atrás
  \item Ctrl+p: Como la flecha hacia arriba
  \item Ctrl+n: Como la flecha hacia abajo
  \item Ctrl+b: Como la flecha hacia la izquierda
  \item Ctrl+f: Como la flecha hacia la derecha
\end{itemize}

\end{document}

